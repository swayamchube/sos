\begin{definition}[Homomorphism]
	A \textit{homomorphism} $\phi:G\to\overline{G}$ is a mapping from $G$ into $\overline{G}$ that preserves the group operation; that is, $\phi(ab)=\phi(a)\phi(b)$ for all $a,b\in G$.
\end{definition}

\begin{definition}[Kernel]
	The \textit{kernel} of a homomorphism $\phi:G\to\overline{G}$ is the set 
	$$\Ker(\phi)=\left\{x\in G\mid \phi(x)=e_{\overline{G}}\right\}$$
\end{definition}

In simple words, the concept of a \textit{homomorphism} generalizes the concept of an \textit{isomorphism} to general functions that preserve group operations.

\begin{proposition}
	Let $\phi$ be a homomorphism from a group $G$ to a group $\overline{G}$ and let $g\in G$. Then
	\begin{enumerate}
		\item $\phi$ carries the identity of $G$ to the identity of $\overline{G}$
		\item $\phi(g^n)=\phi(g)^n$ for all $n\in\Z$
		\item If $|g|$ is finite, then $|\phi(g)|$ is finite and divides $|g|$
		\item $\Ker\phi$ is a subgroup of $G$.
		\item $\phi(a)=\phi(b)$ if and only if $a\Ker\phi=b\Ker\phi$
		\item If $\phi(g)=g'$, then $\phi^{-1}(g)=\left\{x\in G\mid\phi(x)=g'\right\}=g\Ker\phi$.
	\end{enumerate}
\end{proposition}
\begin{proof}\hfill
	\begin{enumerate}
		\item Trivial
		\item Trivial
		\item Using $(1)$ and $(2)$, we have 
		$$
		e_{\overline{G}} = \phi(e_{G}) = \phi(g^{|g|}) = \phi(g)^{|g|}
		$$
		the conclusion now follows.
		\item It is easy to see that $\Ker\phi$ must be closed. Further, the group operation is associativer on $\Ker\phi$ since it is a subset of $G$. Due to $(1)$, $e_G\in\Ker\phi$. Finally, note that if $\phi(a)=e_{\overline{G}}$, we must have $\phi(a^{-1})=e_{\overline{G}}$. This shows that $\Ker\phi$ is a subgroup of $G$.
		\item Suppose $\phi(a)=\phi(b)$. Then, let $x\in\Ker\phi$. We have
		$$
		\phi(b^{-1}ax) = \phi(b^{-1})\phi(a)\phi(x) = e
		$$
		Thus $b^{-1}ax\in\Ker\phi$ or $ax\in b\Ker\phi$ for all $x\in\Ker\phi$ which is equivalent to $a\Ker\phi\subseteq b\Ker\phi$. Similarly, one can obtain $b\Ker\phi\subseteq a\Ker\phi$.
		\item Follows from $(5)$.
	\end{enumerate}
\end{proof}

\begin{proposition}
	Let $\phi$ be a homomorphism from a group $G$ to a group $\overline{G}$ and let $H$ be a subgroup of $G$. Then
	\begin{enumerate}
		\item $\phi(H)=\{\phi(h)\mid h\in H\}$ is a subgroup of $\overline{G}$
		\item If $H$ is cyclic, then $\phi(H)$ is cyclic
		\item If $H$ is Abelian, then $\phi(H)$ is Abelian
		\item If $H$ is normal in $G$, then $\phi(H)$ is normal in $\phi(G)$
		\item If $|\Ker\phi|=n$, then $\phi$ is an $n$-to-$1$ mapping from $G$ onto $\phi(G)$.
		\item If $|H|=n$, then $|\phi(H)|$ divides $n$.
		\item If $\overline{K}$ is a subgroup of $\overline{G}$, then $\phi^{-1}(\overline{K})$ is a subgroup of $G$.
		\item If $\overline{K}$ is a normal subgroup of $\overline{G}$, then $\phi^{-1}(\overline{K})$ is a normal subgroup of $G$.
		\item If $\phi$ is onto and $\Ker\phi=\{e\}$, then $\phi$ is an isomorphism from $G$ to $\overline{G}$. 
	\end{enumerate}
\end{proposition}
\begin{proof}\hfill
	\begin{enumerate}
		\item This is a routine proof
		\item Routine, again
		\item Trivial.
		\item Let $h\in H$ and $g\in G$. Then, since $H$ is normal in $G$, $ghg^{-1}\in H$. Thus, $\phi(ghg^{-1}) = \phi(g)\phi(h)\phi(g)^{-1}\in\phi(H)$. This completes the proof.
		\item Trivial due to Property 6 of \textbf{Proposition 10.74}
		\item Let $\phi_H$ denote the restriction of $\phi$ to $H$, that is, $\phi_H:H\to\phi(H)$. Then, according to the previous property, we trivially have that $|\phi(H)||\Ker \phi| = |H|$
		\item Trivial using the first subgroup test which was discussed in \textbf{Chapter 3}.
		\item Trivial using \textbf{Proposition 9.62}.
		\item A consequence of (5).
	\end{enumerate}
\end{proof}

\begin{corollary}
	Let $\phi$ be a group homomorphism from $G$ to $\overline{G}$. Then $\Ker\phi$ is a normal subgroup of $G$.
\end{corollary}

\begin{theorem}[Jordan, 1870]
	Let $\phi$ be a group homomorphism from $G$ to $\overline{G}$. Then, the mapping from $G/\Ker\phi$ to $\phi(G)$ given by $g\Ker\phi\mapsto\phi(g)$ is an isomorphism. In symbols, $G/\Ker\phi\cong\phi(G)$.
\end{theorem}
\begin{proof}
	We shall use $\psi$ to denote the mapping $g\Ker\phi\mapsto\phi(g)$. The fact that $\psi$ is injective follows directly from Property (6) of \textbf{Proposition 10.74}. To show that $\psi$ is operation preserving, note that 
	$$
	\psi(xy\Ker\phi) = \phi(xy) = \phi(x)\phi(y) = \psi(x\Ker\phi)\psi(y\Ker\phi)
	$$
	The surjectivity is trivial and hence, we are done.
\end{proof}

\begin{proposition}
	Every normal subgroup of a group $G$ is the kernel of a homomorphism of $G$. In particular, a normal subgroup $N$ is the kernel of the mapping $\psi:G\to G/N$ which takes $g\mapsto gN$.
\end{proposition}
\begin{proof}
	Let $x\in N$. Note trivially that $xN=N$, since $N$ must be closed under multiplication by $x$ and the multiplication is invertible. Say there exists $y\in G$ such that $\psi(y)=N$. We had already established that $e\in N$ and hence $ye=y\in N$ which completes the proof.
\end{proof}