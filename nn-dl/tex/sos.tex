\documentclass[a4paper, 12pt]{report}

\usepackage{
	amsmath,
	amsthm,
	amssymb,
	xcolor,
	soul, 
	cancel, 
	fancyhdr,
	graphicx,
}
\pagestyle{fancy}
\fancyhead[L]{Abstract Algebra}
\fancyhead[R]{\nouppercase{\leftmark}}
\fancyfoot[C]{\thepage}
\usepackage[utf8]{inputenc}
\usepackage[inline]{enumitem}
\usepackage{hyperref}
\hypersetup{
	colorlinks = true,
	linkcolor = black,
}
\usepackage{geometry}
\geometry{
	left = 1in,
	right = 1in,
	top = 1in,
	bottom = 1in,
}
\usepackage[framemethod=tikz]{mdframed}
\usepackage{tikz}

\newcounter{theorem}
\newenvironment{theorem}[1][]{
	\stepcounter{theorem}
	\ifstrempty{#1}{
		\begin{mdframed}[linewidth=1pt,linecolor=blue!10, backgroundcolor=blue!10,]
		\bfseries Theorem~\thechapter.\thetheorem.\normalfont
	}{
		\begin{mdframed}[linewidth=1pt,linecolor=blue!10, backgroundcolor=blue!10,]
		\bfseries Theorem~\thechapter.\thetheorem~(#1).\normalfont
	}
}{\end{mdframed}}
\newenvironment{definition}[1][]{
	\stepcounter{theorem}
	\ifstrempty{#1}{
		\begin{mdframed}[linewidth=1pt, linecolor=gray!15, backgroundcolor=gray!15,]
		\bfseries Definition~\thechapter.\thetheorem.\normalfont
	}{
		\begin{mdframed}[linewidth=1pt, linecolor=gray!15, backgroundcolor=gray!15,]
		\bfseries Definition~\thechapter.\thetheorem~(#1).\normalfont
	}
}{\end{mdframed}}

\newenvironment{proposition}[1][]{
	\stepcounter{theorem}
	\ifstrempty{#1}{
		\begin{mdframed}[linewidth=1pt, linecolor=blue!10, backgroundcolor=blue!10,]
		\bfseries Proposition~\thechapter.\thetheorem.\normalfont
	}{
		\begin{mdframed}[linewidth=1pt, linecolor=blue!10, backgroundcolor=blue!10,]
		\bfseries Proposition~\thechapter.\thetheorem~(#1).\normalfont
	}
}{\end{mdframed}}

\newcounter{corollary}
\newenvironment{corollary}{
	\stepcounter{corollary}
	\begin{mdframed}[linecolor=blue!10, backgroundcolor=blue!10, linewidth=1pt,]
	\bfseries Corollary~\thechapter.\thecorollary.\normalfont
}{\end{mdframed}}

\newenvironment{lemma}[1][]{
	\stepcounter{theorem}
	\ifstrempty{#1}{
		\begin{mdframed}[linewidth=1pt, linecolor=yellow!30, backgroundcolor=yellow!30,]
		\bfseries Lemma~\thechapter.\thetheorem.\normalfont
	}{
		\begin{mdframed}[linewidth=1pt, linecolor=yellow!30, backgroundcolor=yellow!30,]
		\bfseries Lemma~\thechapter.\thetheorem~(#1).\normalfont
	}
}{\end{mdframed}}

\newenvironment{example}[1][]{
	\ifstrempty{#1}{
		\begin{mdframed}[linewidth=1pt, linecolor=red!10, backgroundcolor=red!20,]
		\bfseries Example.\normalfont
	}{
		\begin{mdframed}[linewidth=1pt, linecolor=red!10, backgroundcolor=red!20,]
		\bfseries Example~(#1).\normalfont
	}
}{\end{mdframed}}

\newenvironment{exercise}{
	\begin{mdframed}[linewidth=1pt, linecolor=green!10, backgroundcolor=green!10,]
	\bfseries Exercise.\normalfont
}{\end{mdframed}}

\title{
	{Abstract Algebra}\\
	\small Reading Project: Summer of Science\\
	Indian Institute of Technology Bombay\\
}
\author{Swayam Chube\\\textbf{Mentor:} Eeshan Jain}
\date{Last Compiled: \today}

\renewcommand{\qedsymbol}{$\blacksquare$}
\newcommand{\N}{\mathbb{N}}
\newcommand{\Z}{\mathbb{Z}}
\newcommand{\Q}{\mathbb{Q}}
\newcommand{\R}{\mathbb{R}}
\newcommand{\F}{\mathbb{F}}
\newcommand{\Pb}{\mathbb{P}}

\begin{document}
	\maketitle
	\tableofcontents
	\chapter{Universal Approximation and Barron's Theorem}
	\section{Universal Approximation}
A \textit{deep network} may be represented by the following function:
\begin{equation*}
	f(x; w) = \sigma_L(w_L\sigma_{L-1}(\ldots\sigma_1(w_1x + b_1)\ldots))
\end{equation*}
where each $\sigma_i$ is a vectorized \textit{non-linear activation function}. Some examples are the Rectified Linear Unit (ReLU) activation function given by $\max\{0,x\}$ and the Sigmoid function given by $\frac{1}{1 + e^{-x}}$.

We define the \textit{depth} as $L$ and the \textit{width} as $\max_{i}d_i$.

We shall show in this section that \textit{all continuous functions} can be expressed with deep networks to sufficient accuracy.

\begin{definition}[Universal Approximator]
	A class of functions $\mathcal{H}$ is a \textit{universal approximator} if for any continuous function $g$ and compact domain $D$ and any $\varepsilon > 0$, there is $f\in\mathcal{H}$ such that 
	\begin{equation*}
		|f(x) - g(x)|\le\varepsilon \qquad \forall x\in D
	\end{equation*}
\end{definition}

The following theorem helps characterize universal approximators:
\begin{theorem}
	Suppose $\mathcal{H}$ satisfies the following propositions:
	\begin{enumerate}
		\item each $f\in H$ is continuous over $D$ 
		\item for all $x\in D$, there is $f\in\mathcal{H}$ with $f(x)\ne0$
		\item for all $x\ne x'\in D$, there is $f\in\mathcal{H}$ with $f(x)\ne f(x')$
		\item $H$ is closed under $\times$ and vector space operations
	\end{enumerate}
	then $\mathcal{H}$ is a universal approximator.
\end{theorem}

A major theorem related to this is due to \href{https://www.cs.cmu.edu/~epxing/Class/10715/reading/Kornick_et_al.pdf}{Hornik et al}.
\begin{theorem}[Hornik et al. 1989]
	For any continuous activation function $\sigma\colon\R\to\R$ with 
	\begin{equation*}
		\lim_{z\to-\infty}\sigma(z) = 0 \qquad \lim_{z\to\infty}\sigma(z) = 1
	\end{equation*}
	the depth two feedforward netwroks of unbounded width are universal approximators.
\end{theorem}

\begin{theorem}[Park et al.]
	For any square integrable function $f:\R^d\to\R$ and any $\varepsilon > 0$, there is a width $d + 1$ network with ReLU activations such that 
	\begin{equation*}
		\int|f(x) - \hat{f}(x)|^2~dx\le\varepsilon
	\end{equation*}
\end{theorem}

\section{Barron's Theorem}

\end{document}