In the previous chapter, we tried to develop the notion of a group in an informal way. In this chapter, we would like to formalize that same notion. For this, we begin with the definition of a binary operation
\begin{definition}[Binary Operation]
	Let $G$ be a set. A binary operation $\circ$ is simply a function~$\circ:G\times G\to G$ 
\end{definition}

Equipped with the above definition, we are now ready to define a group
\begin{definition}
	Let $G$ be a set and $\circ$ be a binary operation on $G$, then the ordered pair $(G,\circ)$ is said to be a group if the following conditions are satisfied:
	\begin{enumerate}
		\item\textit{Associativity.} For any three (not necessarily distinct) elements $a,b,c\in G$, $a\circ(b\circ c)=(a\circ b)\circ c$.
		\item\textit{Identity.} There exists an element $e$ in $G$ such that $a\circ e=a=e\circ a$ for all elements $a\in G$.
		\item\textit{Inverses.} For all elements $a\in G$, there exists $b\in G$ such that $a\circ b=e=b\circ a$.
	\end{enumerate}
\end{definition}

By the way, note that the definition of a binary operation itself implies that the group is closed, so we need not provide a separate axiom for \textit{closure}.\\

\textbf{Examples of Groups}
\begin{enumerate}
	\item $(\Z,+)$, $(\R,+)$, $(\Q,+)$
	\item $(\R\backslash\{0\},\times)$ and $(\Q\backslash\{0\},\times)$
	\item $(D_n,\circ)$ where $\circ$ is the binary operator for composition of symmetries.
	\item $(GL(n,\R),\times)$ where $GL(n,\R)$ is the set of invertible square matrices of order $n$ with real entries and $\times$ is matrix multiplication. This group is called the \textit{general linear group} of $n\times n$ matrices over $\R$.
\end{enumerate}

\textbf{Examples of non-Groups}
\begin{enumerate}
	\item $(\N,+)$. Since there is no identity in $\N$.
	\item $(\R,\times)$ and $(\Q,\times)$. Since $0$ does not have an inverse.
\end{enumerate}

\begin{exercise}
	Let $U(n)$ denote the set of all natural numbers, less than or equal to $n$ which are coprime to $n$. Show that $(U(n),\cdot)$ where $\cdot$ is multiplication modulo $n$ forms a group.
\end{exercise}
\begin{proof}
	Verifying closure is trivial. Further, it is easy to verify that $1\in U(n)$ is the required identity. Finally, we are left with proving the existance of inverses. Let $a\in U(n)$. Recall, from Bezout's Lemma, there must exist integers $x$ and $y$ such that 
	$$
	ax+ny=\gcd(a,n) = 1
	$$
	Reducing the above equation modulo $n$, we obtain (for some $x'>0$)
	$$
	ax'\equiv1\pmod n
	$$
	The above equation also implies that $\gcd(x',n)=1$, and thus $x'\in U(n)$ and is an inverse of $a$. Since $(U(n),\cdot)$ satisfies the group axioms, it is indeed a group. Further, since multiplication is commutative modulo $n$, the group is also \textit{Abelian}.
\end{proof}


Now that we have seen sufficient examples of groups, we shall have a look at some general properties of groups.\\

Here onwards, for some binary operation $\circ$, $a\circ b$ shall be represented simply by $ab$.

\begin{proposition}[Uniqueness of Identity]
	Let $(G,\circ)$ be a group. Then the identity element of the group is unique.
\end{proposition}
\begin{proof}
	Since $(G,\circ)$ is known to be a group, it has atleast one identity. Suppose, there exist two identies $e_1$ and $e_2$. Then, according to the definition of the identity,
	$$
	e_1 = e_1e_2 = e_2e_1 = e_2
	$$
	This implies that the identity is unique.
\end{proof}
\begin{proposition}[Left and Right Cancellation]
	Let $(G,\circ)$ be a group. Let $a,b,c\in G$.
	\begin{itemize}
		\item If $ba=ca$, then $b=c$
		\item If $ab=ac$, then $b=c$
	\end{itemize}
\end{proposition}
\begin{proof}
	Proving one of them is sufficient, since the proof of the other follows similarly. We have 
	$$
	ba = ca \Longrightarrow  \underbrace{(ba)a^{-1} =(ca)a^{-1} \Longrightarrow b(aa^{-1})=c(aa^{-1})}_{\text{due to associativity}} \Longrightarrow b=c
	$$
\end{proof}
\begin{proposition}[Uniqueness of Inverse]
	Let $(G,\circ)$ be a group. For every $a\in G$, there exists a unique inverse $b\in G$ for the aforementioned $a$.
\end{proposition}
\begin{proof}
	Since $G$ is known to be a group, $a$ has atleast one inverse. Suppose there exist two inverses $b$ and $c$ for $a$. Then, 
	$$
	b = be = b(ac) = (ba)c = c
	$$
	This implies that the inverse of $a$ is unique.
\end{proof}

\begin{proposition}
	Let $a,b\in G$. Then $(ab)^{-1}=b^{-1}a^{-1}$
\end{proposition}
\begin{proof}
	Note that 
	$$
	e = aa^{-1} = (ae)a = (a(bb^{-1}))a^{-1} = ((ab)b^{-1})a^{-1} = (ab)(b^{-1}a^{-1})
	$$
	and due to the uniqueness of the inverse, we have the desired conclusion.
\end{proof}







