We would first begin with some definitions and notations
\begin{definition}
	For any set $S=\{a,b,\cdots\}$, we create a new set $S^{-1} = \{a^{-1},b^{-1},\cdots\}$ by replacing each $x$ in $S$ by $x^{-1}$. Define the set $W(S)$ to be the collection of all formal finite strings of the form $x_1x_2\cdots x_k$ where each $x_i\in S\bigcup S^{-1}$. The elements of $W(S)$ are called \textit{words} from $S$. The empty word is denoted by $e$.\\
	For $w_1$ and $w_2$ in $W(S)$, define $w_1w_2$ as the concatenated word.
\end{definition}

\begin{definition}[Equivalence Classes of Words]
	For any pair of elements $u$ and $v$ of $W(S)$, we say that $u$ is related to $v$ if $v$ can be obtained from $u$ by a finite sequence of insertions or deletions of words of the form $xx^{-1}$ or $x^{-1}x$ where $x\in S$.
\end{definition}

\begin{proposition}
	Let $S$ be a set of distinct symbols. For any word $u$ in $W(S)$, let $\overline{u}$ denote the set of all words in $W(S)$ equivalent to $u$. Then, the set of all equivalence classes of elements of $W(S)$ is a group under the operation $\overline{u}\cdot\overline{v}=\overline{uv}$.
\end{proposition}

The above group is called a Free Group.

\begin{proposition}[Universal Mapping Property]
	Every group is a homomorphic image of a free group.
\end{proposition}
\begin{proof}
	Let $G$ be a group and let $S$ be a set of generators for $G$ and take $F$ to be the free group on $S$. Now consider the mapping 
	$$
	\phi(\overline{x_1x_2\cdots x_n}) = (x_1x_2\cdots x_n)_G
	$$
	It is not hard to see that $\phi$ is well defined. As for operation preserving, we have 
	\begin{align*}
		\phi(\overline{x_1x_2\cdots x_n})\phi(\overline{y_1y_2\cdots y_m}) &= (x_1x_2\cdots x_n)_G (y_1y_2\cdots y_m)_G\\
		&= \phi(\overline{x_1x_2\cdots x_ny_1y_2\cdots y_n})
	\end{align*}
	Finally, we can say that $\phi$ is onto $G$ since $S$ generates $G$.
\end{proof}

\begin{corollary}
	Every group is isomorphic to a quotient group of a free group.
\end{corollary}

\begin{definition}
	Let $G$ be a group generated by some subset $A=\{a_1,a_2,\cdots,a_n\}$ and let $F$ be the free group on $A$. Let $W=\{w_1,w_2,\cdots,w_m\}$ be a subset of $F$ and let $N$ be the smallest normal subgroup of $F$ containing $W$. We say that $G$ is given by the generators $a_1,a_2,\cdots,a_n$ and the relations $w_1=w_2=\cdots=w_m=e$ if there is an isomorphism from $F/N$ onto $G$ that carries $a_iN$ to $a_i$.
\end{definition}

The notation for the above situation is 
\begin{equation*}
	G = \langle a_1,a_2,\cdots,a_n \mid w_1=w_2=\cdots=w_m=e\rangle
\end{equation*}

\begin{theorem}[Dyck, 1882]
	Let 
	\begin{equation*}
		G = \langle a_1,a_2,\cdots,a_n \mid w_1=w_2=\cdots=w_m=e \rangle
	\end{equation*}
	and let 
	\begin{equation*}
		\overline{G} = \langle a_1,a_2,\cdots,a_n \mid w_1=w_2=\cdots=w_m=w_{m+1}=\cdots=w_{m+k} = e\rangle
	\end{equation*}
	Then $\overline{G}$ is a homomorphic image of $G$.
\end{theorem}
\begin{proof}
	Let $F$ be the free group on $\{a_1,a_2,\cdots,a_n\}$ and let $N$ and $M$ be the smallest normal subgroups containing $\{w_1,w_2,\cdots,w_m\}$ and $\{w_1,w_2,\cdots,w_{m+k}\}$. Then $F/N\cong G$ while $F/M\cong\overline{G}$. Finally if we take the obvious homomorphism $aN\mapsto aM$, it would induce a homomorphism from $G$ to $\overline{G}$.
\end{proof}

\begin{proposition}
	Any group generated by a pair of elements of order 2 is dihedral.
\end{proposition}
\begin{proof}
	Say $G$ is generated by $a$ and $b$, both with order $2$. We divide this into two cases, first, when $|ab|$ is not finite. In this case, we would like to show that $G$ is isomorphic to $D_\infty$. We know due to Dyck that $G$ is isomorphic to a factor group of $D_\infty$, say $D_\infty/H$. Let $h\in H$ and $h\neq e$. Without loss of generality, we may assume that $h$ is of the form $(ab)^i$ or $(ab)^ia$. \\
	If $h = (ab)^i$, then,
	\begin{equation*}
		 H = (ab)^iH = (abH)^i
	\end{equation*}
	\begin{equation*}
		(aH)(abH)(aH) = baH = (abH)^{-1}
	\end{equation*}
	implying that
	\begin{equation*}
		D_\infty/H = \langle aH, bH\rangle = \langle aH, abH\rangle
	\end{equation*}
	Which would imply that $G$ is finite which is not possible.\\
	Next,  if $h=(ab)^ia$, 
	\begin{equation*}
		H = (ab)^iaH = (ab)^iHaH 
	\end{equation*}
	which implies that 
	\begin{equation*}
		(abH)^i = (ab)^iH = (aH)^{-1} = a^{-1}H = aH
	\end{equation*}
	Thus,
	\begin{equation*}
		\langle aH, bH\rangle = \langle aH, abH\rangle\subseteq\langle abH\rangle
	\end{equation*}
	But since $|abH|$ is finite, it would mean $D_\infty/H$ is finite, which would force $H$ to contain only the identity and $G$ to be isomorphic to $D_\infty$.\\
	Finally, if $|ab|$ is finite, we would be able to write $G = \langle a,b\rangle = \langle a, ab\rangle$. And finally, one can show that $a$ and $ab$ satisfy the relations for $D_n$ and we are done.
\end{proof}